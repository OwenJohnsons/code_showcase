\begin{abstract}
    The investigation aimed to explore Compton scattering phenomenon and relate it in a relativistic context. This was done in two separate investigations. The first investigation demonstrated that gamma rays' energy had a linear relationship to the angle in which they were scattered. It was also shown that the experimental data was in-line with theoretical predictions. This was then further expanded on by comparing the differential cross-section of the experimental data to the Klein–Nishina prediction this also yielded a result in line with theoretical models with an average difference of 1.4\% on each data point. The second investigation explored Compton scattering in a relativistic sense. This was done with a multitude of radioactive sources to compare the electron's rest energy. It was found that when this was determined in a non-relativistic context, a linear relationship was observed with the y-intercept of $0.514 \pm 0.004$ MeV—calculating the rest energy in a relativistic context produced a result of $0.515 \pm 0.007$ MeV.  It was further demonstrated by the dependence that energy and momentum had on the velocity of the electron.
\end{abstract}