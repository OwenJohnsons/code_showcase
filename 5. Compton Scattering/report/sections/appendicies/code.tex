\newpage 

\section{Relevant Code Used} \label{code} 

\begin{lstlisting}[language = Python, frame = single, caption = {Package Import}]
import numpy as np 
import pandas as pd 
import math
import uncertainties
from uncertainties import ufloat_fromstr
from uncertainties import unumpy as unp
import matplotlib.mlab as mlab
import matplotlib.pyplot as plt
from scipy import stats
from scipy . optimize import curve_fit
from scipy . interpolate import interp1d
from scipy import optimize

# --- plot parameters --- 
plt.rcParams["figure.figsize"] = (15,10)
plt.rc('font', family = 'serif', serif = 'cmr10') 
plt.rcParams.update({'font.size': 22})
\end{lstlisting}

\begin{lstlisting}[language = Python, frame = single, caption = {Importing data}]
    # --- Importing data from .csv -- 

    # --- Importing data from table 1 provided in raw data word document ---  
    
    part_1 = pd.read_csv('data\Compton_scattering_formula_and_differential_cross_section.csv')
    
    part_2 = pd.read_csv('data\compton_scattering_electron_mass_rel.csv')
\end{lstlisting}

\subsection{Code used for evaluation of scattered gamma ray energies and cross section.}

\begin{lstlisting}[language = Python, frame = single, caption = {Variable Defnition for data analysis}]
    # --- Variable Definition --- 

    # --- Part 1 --- 
    
    angle = part_1['Angle']
    counts_time = np.array(part_1['Counting_time'])
    gamma_ray_energy = part_1['scattered_gamma_ray_energy']
    gamma_ray_uncer = part_1['scattered_gamma_ray_energy_uncer']
    photopeak_area = part_1['photopeak_net_area']
    photopeak_uncer = part_1['photopeak_net_area_uncer']
    
    # --- conversion and uncertainty coupling --- 
    
    
    
    angle = unp.uarray(angle, 1)
    gamma_rays = unp.uarray(gamma_ray_energy, gamma_ray_uncer)*0.001 # conversion from KeV to MeV 
    photopeak = unp.uarray(photopeak_area, photopeak_uncer) 
    
    print(photopeak)
    
    angle_r = angle*(np.pi/180)
\end{lstlisting}

\begin{lstlisting}[language = Python, frame = single, caption = {Function defintion}]
    # -- function definition --- 

    def lin_func(x, m, c):
        return m*x + c #equation of a line y = mx + c 
    
    def exp(x, a, b):
        return a*np.exp(-b*x)
    
    # - differential cross section - 
    
    def Klein_Nishina(theta):
        r = 2.82e-13
        alpha = 1.29
        A = (r**2/2)*((1 + np.cos(theta)**2)/(1 + alpha*(1 - np.cos(theta)**2)))
        t = (alpha**2)*(1 - np.cos(theta))**2
        b = (1 + np.cos(theta)**2)*(1 + alpha*(1 - np.cos(theta)))
        B = t/b
        return A*B 
    
    
    def gamma_sum(energy, photopeak, counting_time):
        peak_eff = .1522*(energy)**-1.1325
        return (photopeak/counting_time)/peak_eff
        
    
    def diff_cross(gamma_sum):
        t = 42.583 # time in years from August of 1977 to March 2020, when data was taken. 
        I = 1.013e6*np.exp(-t/43.48) 
        delta_omega = np.pi/26**2
        N = ((79.3)*(6.0221409e+23)*(13))/27
        return gamma_sum/(I*delta_omega*N)
    
    def KleinNishina (ang) : #in cm squared / steradians
        alpha = 1.29
        r = 2.82e-13
        return ((r**2) /2)*((1+(np.cos(ang))**2)/(1+((alpha)*(1 - np.cos(ang))))**2)*(1+( (alpha**2)*((1 - np.cos(ang))**2) /(  (np.cos(ang))**2)*(1+((alpha)*(1 - np.cos(ang))))))
\end{lstlisting}

% \begin{lstlisting}[language = Python, frame = single]
%     # --- setting up data for graphing  -- 

%     y = 1/unp.nominal_values(gamma_rays)
%     yerr = unp.std_devs(gamma_rays)  
    
%     x = 1 - np.cos(unp.nominal_values(angle_r))
%     xerr = unp.std_devs(angle_r)
    
%     x_fit = np.linspace(0, x.max(), 10)
    
%     # --- fitting data for graph --- 
    
%     popt1, pcov1 = curve_fit(lin_func, x, y)
%     fit_data = lin_func(x_fit, popt1[0], popt1[1])
    
%     # --- Theoretical Fit --- 
    
%     th_data = lin_func(x_fit, 1.956, 1.51)
    
%     # --- plotting --- 
    
%     plt.plot(x_fit, th_data, 'blue', label = 'Theoretical Prediction')
%     plt.plot(x_fit, fit_data, 'black', label = 'Data Fit')
%     plt.errorbar(x, y, xerr = xerr, yerr = yerr, capsize= 5, ls='none', color = 'red')
%     plt.scatter(x, y, color = 'red', label = 'Data Points')
    
%     plt.xlabel('$1 − \cos (\phi ))$ (Radians)')
%     plt.ylabel('$1/E_\gamma$ (MeV)')
    
%     plt.grid()
%     plt.legend()
%     plt.savefig('images/angle_vs_gammaray.png')
% \end{lstlisting}

\begin{lstlisting}[language = Python, frame = single]
    # --- Slope of Curve --- 

    print('The slope of the above curve is:', "{:.3f}".format(popt1[0]))
    
    # --- Finding Intercept --- 
    
    x = 0
    y = (popt1[0]*x) + popt1[1]
    print('y-intercept of the line:', "{:.3f}".format(y))
    
    # --- Comparison with theoretical values --- 
    
    print('The slope of the fitted data is', "{:.3f}".format(1.956/popt1[0] - 1),'% greater than predicted value for slope.')
    
    xth = 0
    yth = (1.956*x) + 1.51
    print('y-intercept of the theoretical line is:', "{:.3f}".format(yth))
    
    print('y-intercept of the fitted data is', "{:.3}".format(y/yth - 1),'% greater than predicted value for slope.')
    
    
    # --- Uncetainties on the Slope --- 
    
    
    print('Uncertainty on the y-intercept is given as,', "{:.3}".format(np.sqrt(pcov1[0][0])))
    print('Uncertainty on the slope is given as,', "{:.3}".format(np.sqrt(pcov1[1][1])))
    
    
    # --- Final Values are Given by --- 
    
    print("Intercept: ", "{:.3f}".format(y), '+-', "{:.2}".format(np.sqrt(pcov1[0][0])))
    print("Slope: ", "{:.3f}".format(popt1[0]), '+-', "{:.1}".format(np.sqrt(pcov1[1][1])))
    
\end{lstlisting}

\begin{lstlisting}[language = Python, frame = single]
    # --- data for graph --- 

    x = unp.nominal_values(angle_r)
    xerr = unp.std_devs(angle_r)
    
    p1 = gamma_sum(gamma_rays, photopeak, counts_time)
    # print(p1)
    p2 = diff_cross(p1)
    # print(p2)
    
    
    # --- theortical model --- 
    
    yth = KleinNishina(x)
    
    # -- 
    correction_factor = (yth/p2).mean()
    # --
    
    p2 = diff_cross(p1)*correction_factor
    
    y = unp.nominal_values(p2)
    y_err = unp.std_devs(p2)
    
    # --- fitting function --- 
    
    x_s = np.linspace(x.min() - .1, x.max(), 1000)
    
    popt1, pcov1 = curve_fit(exp, x, y, sigma = y_err, absolute_sigma = True, p0 = [1e-26, 2])
    fit_data = exp(x_s, popt1[0], popt1[1])
    
    # --- 
    
    popt2, pcov2 = curve_fit(exp, x, yth, p0 = [7e-26, .7])
    th_data = exp(x_s, popt2[0], popt2[1])
    
    # --- Plotting ---
    
    plt.plot(x_s, th_data, label = 'Theortical Prediction', color = 'blue')
    plt.scatter(x, y, label = 'Data Points ', color = 'red')
    plt.plot(x_s, fit_data, label = 'Corrected Data Fit', color = 'black')
    plt.errorbar(x, y, xerr = xerr, yerr = y_err, capsize= 5, ls='none', color = 'red')
    
    plt.xlabel('Angle $(\phi )$, (Radians)')
    plt.ylabel('Differential Cross Section')
    
    plt.legend()
    plt.grid()
    plt.savefig('images/angle_vs_cross_section.png')
\end{lstlisting}

\begin{lstlisting}[language = Python, frame = single]
    # --- Variable Compile --- 

    x = unp.nominal_values(angle_r)
    xerr = unp.std_devs(angle_r)
    yth = KleinNishina(x)
    y = unp.nominal_values(p2)
    y_err = unp.std_devs(p2)
    percent = (y/yth - 1)*100
    # print(percent)
    
    print(stats.chisquare(f_obs=y, f_exp=yth))
    
    # --- write to .csv --- 
    
    Output = [x, xerr, y*1e26, y_err*1e26, yth*1e26, percent]
    
    # print(Output)
    dataset = pd.DataFrame({'Angle': np.round(Output[0], 3), 'Angle_Error': np.round(Output[1], 3), 'Diff_Cross_Section': np.round(Output[2], 3), 'Diff_Cross_Section_Uncer': np.round(Output[3], 3), 'Diff_Cross_Section_Th': np.round(Output[4], 3), 'Percent': np.round(Output[5], 1)})
    print(dataset)
    
    print(dataset.to_latex(index=False)) 
    dataset.to_csv('data/theor_compare.csv')
\end{lstlisting}

\newpage
\subsection{Code used for investigating relativistic relations} \label{code_p2}

\begin{lstlisting}[language = Python, frame = single]
    # --- Loading Data --- 

    sources = part_2['Source']
    photopeak = part_2['Photopeak']
    photopeak_uncer = part_2['Photopeak_Uncertainty']
    compton_edge = part_2['Compton_Edge']
    compton_edge_uncer = part_2['Compton_Edge_Uncertainty']
    
    # --- converting data and coupling --- 
    
    photopeak = unp.uarray(photopeak, photopeak_uncer)*0.001 
    compton_edge = unp.uarray(compton_edge, compton_edge_uncer)*0.001  
\end{lstlisting}

\begin{lstlisting}[language = Python, frame = single]
    x = unp.nominal_values(photopeak)
    y = unp.nominal_values(compton_edge)
    x_err = unp.std_devs(photopeak)
    y_err = unp.std_devs(compton_edge)
    
    # --- Fitting --- 
    
    popt1, popc1 = curve_fit(lin_func, x, y)
    fit_data = lin_func(x, popt1[0], popt1[1])
    
    # --- Plotting ---
    
    plt.scatter(x, y, color = 'red')
    plt.errorbar(x, y, xerr = x_err, yerr = y_err, capsize= 5, ls='none', color = 'red')
    plt.plot(x, fit_data, color = 'black')
    plt.grid()
    plt.xlabel('Photopeak (MeV)') 
    plt.ylabel('Compton Edge (MeV)') 
    plt.savefig('images/photo_peak_vs_compton_edge.png')
\end{lstlisting}

\begin{lstlisting}[language = Python, frame = single]
    # --- Slope of Curve --- 

    print('The slope of the above curve is:', "{:.3f}".format(popt1[0]))
    
    # --- Finding Intercept --- 
    
    x = 0
    y = (popt1[0]*x) + popt1[1]
    print('y-intercept of the line:', "{:.3f}".format(y))
\end{lstlisting}

\begin{lstlisting}[language = Python, frame = single]
    def classical_energy(gamma_energy, compton_edge):
    return ((2*gamma_energy - compton_edge)**2)/(2*compton_edge)

def rel_energy(gamma_energy, compton_edge): 
    return 2*gamma_energy*(gamma_energy - compton_edge)/compton_edge
\end{lstlisting}

\begin{lstlisting}[language = Python, frame = single]
    x = unp.nominal_values(compton_edge)
    x_err = unp.std_devs(compton_edge)
    
    clre = classical_energy(photopeak, compton_edge)
    
    y = unp.nominal_values(clre)
    y_err = unp.std_devs(clre)
    
    # --- Fitting --- 
    
    popt1, popc1 = curve_fit(lin_func, x, y)
    fit_data = lin_func(x, popt1[0], popt1[1])
    
    plt.plot(x, fit_data, color = 'black')
    plt.scatter(x, y, color = 'red')
    plt.errorbar(x, y, xerr = x_err, yerr = y_err, capsize= 5, ls='none', color = 'red')
    plt.xlabel('T (MeV)')
    plt.ylabel('$mc^2$ (MeV)')
    plt.grid()
    plt.savefig('images/electron_cl.png')
\end{lstlisting}

\begin{lstlisting}[language = Python, frame = single]
    # --- Slope of Curve --- 

    print('The slope of the above curve is:', "{:.3f}".format(popt1[0]))
    
    # --- Finding Intercept --- 
    
    x = 0
    y = (popt1[0]*x) + popt1[1]
    print('y-intercept of the line:', "{:.3f}".format(y))
\end{lstlisting}

\begin{lstlisting}[language = Python, frame = single]
    x = unp.nominal_values(compton_edge)
    x_err = unp.std_devs(compton_edge)
    
    relre = rel_energy(photopeak, compton_edge)
    ovr_err = relre.mean()
    
    y = unp.nominal_values(relre)
    y_err = unp.std_devs(relre)
    
    # --- Accepted Value of rest mass --- 
    #0.5110 MeV
    
    plt.axhline(y = 0.511, color = 'green', alpha = .7, label = 'Experimentally Accepted Value')
    
    # --- comparing to average values --- 
    
    plt.axhline(y = y.mean(), color = 'purple', alpha = .7, label = 'Average Value of Data')
    
    plt.scatter(x, y, color = 'red')
    plt.errorbar(x, y, xerr = x_err, yerr = y_err, capsize= 5, ls='none', color = 'red')
    plt.grid()
    plt.xlabel('T (MeV)')
    plt.ylabel('$mc^2$ (MeV)')
    plt.legend()
    plt.savefig('images/electron_rel.png')
\end{lstlisting}

\begin{lstlisting}[language = Python, frame = single]
    print('Average rest energy', '{:.3f}'.format(ovr_err), 'MeV')
    print('The difference in rest energy between experimental data and the accepted value is', '{:.3f}'.format(y.mean() - 0.511), 'MeV, this is a differnce of', '{:.3f}'.format(0.511/y.mean()), '%')
    print('Errors on the y-values',  '{:.3f}'.format(y_err.mean()))
\end{lstlisting}

\begin{lstlisting}[language = Python, frame = single]
    def electron_velocity(E, T):
    return (T*(2*E - T))/(T**2 - 2*E*T + 2*E**2)

def relativistic_factor(E, T):
    return 1 + ((T**2)/(2*E*(E - T)))

def total_energy(E, T):
    return (T**2 - 2*T*E + 2*E**2)/(T)

def momentum(E, beta):
    return E*beta

def unique_fit(x, a, b, c):
    return 1/np.sqrt(a + b*x**2) + c
\end{lstlisting}

\begin{lstlisting}[language = Python, frame = single]
    beta = electron_velocity(photopeak, compton_edge)
    gamma = relativistic_factor(photopeak, compton_edge)
    E = total_energy(photopeak, compton_edge)
    pc = momentum(E, beta)
    
    
    # --- Writing data to CSV --- 
    
    Output = [sources, unp.nominal_values(beta), unp.std_devs(beta), unp.nominal_values(gamma), unp.std_devs(gamma), unp.nominal_values(clre), unp.std_devs(clre), unp.nominal_values(relre), unp.std_devs(relre), unp.nominal_values(E), unp.std_devs(E)]
    
    # print(Output)
    dataset = pd.DataFrame({'Source': Output[0], 'beta_values': np.round(Output[1], 3), 'beta_uncertainties': np.round(Output[2], 3), 'Gamma_values': np.round(Output[3], 3), 'Gamma_uncertainties': np.round(Output[4], 3), 'Classical_Rest_Energy_MeV': np.round(Output[5], 3), 'Classical_Rest_Energy_uncertainties_MeV': np.round(Output[6], 3), 'Relativistic_Rest_Energy_MeV': np.round(Output[7], 3), 'Relativistic_Rest_Energy_uncertainties_MeV': np.round(Output[8] , 3), 'Total_Energy': np.round(Output[9], 3), 'Total_Energy_Uncer': np.round(Output[10], 3)})
    # print(dataset)
    print(dataset.to_latex(index=False))
    dataset.to_csv('data/relativistic_relations.csv')
\end{lstlisting}

\begin{lstlisting}[language = Python, frame = single]
    x = unp.nominal_values(beta)
    y = unp.nominal_values(pc)
    x_err = unp.std_devs(beta)
    y_err = unp.std_devs(pc)
    
    plt.ylabel('Momentum (pc)')
    plt.xlabel('Electron Velocity $( \\beta )$')
    
    # --- fitting function --- 
    
    x_s = np.linspace(0, x.max(), 1000)
    
    popt1, pcov1 = curve_fit(unique_fit, x, y, sigma = y_err, absolute_sigma = True, p0 = [.1, 3, 1])
    fit_data = unique_fit(x_s, popt1[0], popt1[1], popt1[2])
    
    plt.plot(x_s, fit_data, color = 'black')
    plt.scatter(x, y, color = 'red')
    plt.errorbar(x, y, xerr = x_err, yerr = y_err, capsize= 5, ls='none', color = 'red')
    plt.grid()
    plt.savefig('images/pc_vs_Velocity')
\end{lstlisting}

\begin{lstlisting}[language = Python, frame = single]
    x = unp.nominal_values(beta)
    y = unp.nominal_values(gamma)
    x_err = unp.std_devs(beta)
    y_err = unp.std_devs(gamma)
    
    # --- fitting function --- 
    
    x_s = np.linspace(0, x.max(), 1000)
    
    popt1, pcov1 = curve_fit(unique_fit, x, y, sigma = y_err, absolute_sigma = True, p0 = [.1, 3, 1])
    fit_data = unique_fit(x_s, popt1[0], popt1[1], popt1[2])
    
    plt.plot(x_s, fit_data, color = 'black')
    
    plt.scatter(x, y, color = 'red')
    plt.errorbar(x, y, xerr = x_err, yerr = y_err, capsize= 5, ls='none', color = 'red')
    
    plt.grid()
    plt.ylabel('Relativistic Factor $(\gamma)$')
    plt.xlabel('Electron Velocity $( \\beta )$')
    plt.savefig('images/RelFactor_vs_Velocity')
\end{lstlisting}

\begin{lstlisting}[language = Python, frame = single]
    x = unp.nominal_values(beta)
    y = unp.nominal_values(E)
    x_err = unp.std_devs(beta)
    y_err = unp.std_devs(E)
    
    # --- fitting function --- 
    
    x_s = np.linspace(0, x.max(), 1000)
    
    popt1, pcov1 = curve_fit(unique_fit, x, y, sigma = y_err, absolute_sigma = True, p0 = [.1, 3, 1])
    fit_data = unique_fit(x_s, popt1[0], popt1[1], popt1[2])
    
    plt.plot(x_s, fit_data, color = 'black')
    
    plt.scatter(x, y, color = 'red')
    plt.errorbar(x, y, xerr = x_err, yerr = y_err, capsize= 5, ls='none', color = 'red')
    
    plt.grid()
    plt.ylabel('Total Energy (E) (MeV)')
    plt.xlabel('Electron Velocity $( \\beta )$')
    plt.savefig('images/TE_vs_Velocity')
\end{lstlisting}

